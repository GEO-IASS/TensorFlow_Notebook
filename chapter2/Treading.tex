\section{线程和队列}
\begin{quote}
	\emph{注意在TensorFlow1.2之前我们推荐用多线程,队列输入pipeline,在TensorFlow1.2开始我们推荐使用tf.contrib.data模块。tf.contrib.data提供了一个更加简单的结构构建高效的输入pipeline,我们已经停止了之前正在开发的多线程和队列输入pipeline,我们帮依然维护旧的代码的开发者维护文档。}
\end{quote}
%多线程及时是一个强有力的而且广泛使用的支持一步计算的机制,入上面章节所述,TensorFlow计算图节点和计算图实现,一个队列是一个已经准备好的节点,向一个变量;以它节点可以修改它的内容。类似的,节点可以入队新的元素到队列或者从队列中出去已近存在的元素,TensorFlow队列提供了一个方法结合多个计算步。如果队列为空依然出队或者队列已满依然入队豆浆产生阻塞,挡这两个条件不满足的时候阻塞解除基础处理。TensorFlow实现了多个队列类,不同的类之间的不同是元素从队列移除的顺序。为了更好的理解我们考虑一个简单的情况我们创建"First in first out"队列然后填充0.然后我们构造一张图从队列中获取元素增加1,然后将它放在队列的尾部,慢慢地队列中的数字增加。
\begin{python}
sess = tf.Session()
q = tf.FIFOQueue(3,"float")
init = q.enqueue_many(([0.,0.,0.],))
x = q.dequeue()
y = x+1
q_inc = q.enqueue([y])
init.run(session=sess)
q_inc.run(session=sess)
q_inc.run(session=sess)
q_inc.run(session=sess)
q_inc.run(session=sess)
\end{python}
Enqueue,EnqueueMany和Dequeue是一个特别的节点。它们是指向队列真实值的指针,允许它们改变状态。我们推荐你考虑这些操作的时候用面向对象的理解,事实上在Python API中这些操作通过调用队列的方法。
\begin{quote}
	\emph{注意Queue方法必须运行在相同的设备上,不兼容的设备放置将在创建这些操作的时候被忽略}
\end{quote}
\subsection{队列用法}
像tf.FIFOQueue和tf.RandomShuffleQueue是在图上执行异步计算的重要的TensorFlow对象。典型的队列输入pipline用RandomShuffleQueue为训练模型准备输入:
\begin{itemize}
	\item 多线程准备训练数据和将数据入队
	\item 训练线程执行训练操作从队列出队mini-batch
\end{itemize}
我们推荐使用Dataset的shuffle和batch方法完成这个任务。然而,如果你仍然愿意使用队列版本,你可以在tf.train.shuffle\_batch中找到完美的实现。

下面展示一个简单的实现,这个函数获取一个source tensor,capacity和batch size作为参数返回一个批量打乱的出队tensor。
\begin{python}
def simple_shuffle_batch(source, capacity, batch_size=10):
  # Create a random shuffle queue.
    queue = tf.RandomShuffleQueue(capacity=capacity,
                                  min_after_dequeue=int(0.9*capacity),
                                  shapes=source.shape, dtypes=source.dtype)

    # Create an op to enqueue one item.
    enqueue = queue.enqueue(source)

    # Create a queue runner that, when started, will launch 4 threads applying
    # that enqueue op.
    num_threads = 4
    qr = tf.train.QueueRunner(queue, [enqueue] * num_threads)

    # Register the queue runner so it can be found and started by
    # `tf.train.start\_queue\_runners` later (the threads are not launched yet).
    tf.train.add_queue_runner(qr)

    # Create an op to dequeue a batch
    return queue.dequeue_many(batch_size)
\end{python}
当tf.train.start\_queue\_runners开始的时候,或者直接通过tf.train.MonitoredSession,QueueRunner将在后台开启进程填充队列,同时主线程执行dequeue\_many操作从中拉取数据,现在这些操作不相互依赖,除非间接地通过队列的内部依赖。简单的用这个函数像这样:
\begin{python}
# create a dataset that counts from 0 to 99
input = tf.constant(list(range(100)))
input = tf.contrib.data.Dataset.from_tensor_slices(input)
input = input.make_one_shot_iterator().get_next()

# Create a slightly shuffled batch from the sorted elements
get_batch = simple_shuffle_batch(input, capacity=20)

# `MonitoredSession` will start and manage the `QueueRunner` threads.
with tf.train.MonitoredSession() as sess:
    # Since the `QueueRunners` have been started, data is available in the
    # queue, so the `sess.run(get\_batch)` call will not hang.
    while not sess.should_stop():
        print(sess.run(get_batch))
\end{python}
输出
\begin{python}
[ 8 10  7  5  4 13 15 14 25  0]
[23 29 28 31 33 18 19 11 34 27]
[12 21 37 39 35 22 44 36 20 46]
\end{python}
对于更多的情况有tf.train.MonitoredSession提供的自动线程启动和管理是足够的,在极少的情况下不行,TensorFlow提供了手动管理你的线程的工具。
\subsection{手动线程管理}
正如我们看到的,TensorFlow Session是多线程的而且是线程安全的,因此多线程能够容易的在相同的会话和运行操作中使用。然而,不总是很容易实现一个Python程序按照要求驱动线程,所有的线程必须能同时停止,特别是必须捕获和报告,队列停止的时候必须被合适的关闭。TensorFlow提供了两个类:tf.train.Coordinator和tf.train.QueueRunner。这两个类帮助多线程一起停止,向程序报告异常等待它们停止,QueueRunner类被用于创建一个线程协作同一队列中的入队tensor。
\subsection{Coordinator}
tf.train.Coordinator类管理TensorFlow程序的后台线程帮助多线程一起停止,关键的方法是:
\begin{itemize}
	\item tf.train.Coordinator.should\_stop:如果线程应该被停止返回True。
	\item tf.train.Coordinator.request\_stop:请求应该停止的线程。
	\item tf.train.Coordinator.join:等待直到指定的线程被停止。
\end{itemize}
你首先创建一个Coordinator对象然后创建一些用于协调的线程。线程通常循环运行当shold\_stop为True是停止。任何线程都可以决定计算应该被停止。它仅仅必须调用request\_stop(),should\_stop()返回True是其它线程停止。
\begin{python}
# Using Python's threading library.
import threading

# Thread body: loop until the coordinator indicates a stop was requested.
# If some condition becomes true, ask the coordinator to stop.
def MyLoop(coord):
    while not coord.should_stop():
    ...do something...
    if ...some condition...:
        coord.request_stop()

# Main thread: create a coordinator.
coord = tf.train.Coordinator()

# Create 10 threads that run 'MyLoop()'
threads = [threading.Thread(target=MyLoop, args=(coord,)) for i in range(10)]

# Start the threads and wait for all of them to stop.
for t in threads:
  t.start()
coord.join(threads)
\end{python}
显然,coordinator可以管理线程做不同的事。它们不是必须和上面的例子一样。coordinator也支持捕获和报告异常,查看\href{https://www.tensorflow.org/api_docs/python/tf/train/Coordinator}{tf.train.Coordinator}文档查看更多信息。
\subsection{QueueRunner}
tf.train.QueueRunner类创建一些线程重复执行入队操作。这些线程可以用coordinator一起停止,另外一个队列runner将运行一个closer操作,如果在coordinator中的队列被报告异常将关闭队列。你可以用一个队列runner实现下面的架构,首先用一个TensorFlow为输入样本建立一个图,添加操作处理将样本送入队列,添加训练操作从队列出队。
\begin{python}
example = ...ops to create one example...
# Create a queue, and an op that enqueues examples one at a time in the queue.
queue = tf.RandomShuffleQueue(...)
enqueue_op = queue.enqueue(example)
# Create a training graph that starts by dequeueing a batch of examples.
inputs = queue.dequeue_many(batch_size)
train_op = ...use 'inputs' to build the training part of the graph...	
\end{python}
在Python的训练程序中,创建一个QueueRunner将运行一些线程处理入队样本、创建一个Coordinator要求queue runner用coordinator开启它的线程。用coordinator写一个训练循环。
\begin{python}
# Create a queue runner that will run 4 threads in parallel to enqueue
# examples.
qr = tf.train.QueueRunner(queue, [enqueue_op] * 4)

# Launch the graph.
sess = tf.Session()
# Create a coordinator, launch the queue runner threads.
coord = tf.train.Coordinator()
enqueue_threads = qr.create_threads(sess, coord=coord, start=True)
# Run the training loop, controlling termination with the coordinator.
for step in range(1000000):
    if coord.should_stop():
        break
    sess.run(train_op)
# When done, ask the threads to stop.
coord.request_stop()
# And wait for them to actually do it.
coord.join(enqueue_threads)
\end{python}
\subsection{处理异常}
线程通过队列runner启动做的比仅仅运行入队操作要多。它们捕获处理队列生成的异常,包括用于报告队列被关闭的tf.errors.OutOfRangeError异常。一个训练中的程序用一个coordinator必须类似的在主循环中捕获和报告异常。下面是上面训练循环的一个改进的例子:
\begin{python}
try:
    for step in range(1000000):
      if coord.should_stop():
        break
      sess.run(train_op)
except Exception, e:
    # Report exceptions to the coordinator.
    coord.request_stop(e)
finally:
    # Terminate as usual. It is safe to call `coord.request\_stop()` twice.
    coord.request_stop()
    coord.join(threads)
\end{python}
