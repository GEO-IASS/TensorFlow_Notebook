\section{数据集}
Dataset模块允许你从简单的,可重用的片段输入pipeline。例如图像模块的pipeline
也许集合分布式文件系统的数据,随机扰动每场图片,随机融合选中的图片为一个batch来训练,pipeline的text模型能利用元素提取的文本数据,转换他们为查找表embedding的标志符将不同长度的序列放在一起成为一个batch。Dataset API使得处理大型数据,不同数据格式和复杂的转换变得很容易。
一个 Dataset
 API包含两个TebsorFlow抽象。
\begin{itemize}
\item 一个tf,contrib.data.Dataset代表一个元素序列,其中的每个元素代表了一个或者更多的Tensor对象。例如在图像pipeline,一个元素可能是单个的训练样本(一对代表了label和属相数据的tensor)有两种方法创建数据集:
	\begin{enumerate}
		\item 创建一个源(Dataset.from\_tensor\_slices())从一个或者更多的tf.Tensor独享构造数据集。
		\item 应用转换格式从一个或者更多的tf.contrib.data.Dataset对象构造数据集。
	\end{enumerate}
	\item tf.contrib.data.Iterator提供主要的从数据集提取元素的方法当Iterator.get\_next()操作执行的时候从Dataset生成下一个元素,典型的行为作为输入pipeline和你的模型之间的接口。最简单的迭代器(iterator)是"one-shot iterator"它结合了数据集和迭代。更多高级使用Nike一用Iterator.initializer操作用不同的数据集重新初始化和参数化一个迭代器,例如在一个程序多次迭代训练样本和验证集。
\end{itemize}
\subsection{基本的机制}
为了开始一个输入pipeline你需要定义一个源。例如从内存中的一些tensor构造一个数据集。你可以使用tf.contrib.data.Dataset.from\_tensors()或者tf.contrib.data.Dataset.from\_tensor\_slices()。如果你的输入数据在磁盘上,推荐你用TFRecord
格式,你可以构造一个tf.contrib.data.TFRecordDataset。

当你有一个Dataset对象的时候,你可以通过链式方法调用tf.contrib.data.Dataset对象转化成新的Dataset。例如例如你可以用之前的元素转换作为Dataset.map()(应用一个函数到每个元素)多元素转换像Dataset.batch()。查看文档\href{https://www.tensorflow.org/api_docs/python/tf/contrib/data/Dataset}{tf.contrib.data.Dataset}完成列表转换。
最常用的方法是消耗从Dataset来的值是创建一个迭代器对象,提供访问一个元素的数据集的元素一次(调用Dadaset.make\_one\_shot\_iterator()),一个tf.contrib.data.Iterator提供两个操作:Iterator.initializer重新初始化你的迭代器状态;Iterator.get)next()返回迭代器下一个元素的Tensor。
\subsection{数据结构}
一个数据及包含有相同结构的元素。一个元素包含一个或者更多称为组建的的tf.Tensor对象。每个组件有tf.DTpe代表在tensor中元素的数据类型,Dataset.output\_types和Dataset.output\_shapes属性允许你查看每个数据集元素的组建的类型和形状。这些参数的潜逃结构映射元素的结构(也许是一个tensor,一个tensor元组,或者嵌套的tensor元组)
\begin{python}
dataset1 = tf.contrib.data.Dataset.from_tensor_slices(tf.random_uniform([4,10]))
print(dataset1.output_types)#tf.float32
print(dataset1.output_shape)#(10,)
dataset2 = tf.contrib.data.Dadaset.from_tensor_slices(tf.random_uniform([4]),
tf.random_uniform([4,100],maxval=100,dtype=tf.int32))
print(dataset2.output_types)#(tf.float32,tf.int32)
print(dataset2.output_shapes)#((),(100,))
datasets = tf.contrib.data.Dataset.zip((dataset1,dataset2))
print(dataset3.output_types)#(tf.float32,(tf.float32,tf.int32))
print(dataset3.output_shapes)#(10,((),(100,)))
\end{python}
给每个元素的组成名字是很方便的,例如他们代表不同训练样本的特征。另外,你可以用collections.namedtuple或者一个字典映射字符串到tensor代表Dataset的单个元素。
\begin{python}
dataset = tf.contrib.data.Dataset.from_tensor_slices(
	"a":tf.random_uniform([4]),
	"b":tf.random_uniform([4,100],maxval=100,dtype=tf.int32))
print(dataset.output_types())#{'a':tf.float32,'b':tf.int32}
print(dataset.output_shape)#{'a':(),'b':(100,)}
\end{python}
Dataset转换支持任何的数据结构,当你用Dataset.map(),Dataset.flat\_map和Dataset.filter()转换应用函数到每个元素,元素结构决定函数的参数:
\begin{python}
dataset1 = dataset1.map(lambda x: ...)
dataset2 = dataset2.flat_map(lambda x, y: ...)
# Note: Argument destructuring is not available in Python 3.
dataset3 = dataset3.filter(lambda x, (y, z): ...)
\end{python}
\subsection{创建一个迭代器}
当你创建一个Dataset代表你的输入数据的时候,下一步是创建一个迭代器从数据集中访问元素,Dataset API当前支持三种迭代器:
\begin{itemize}
	\item one-shot
	\item initilizable
	\item reinitilizable
	\item feedable
\end{itemize}
one-shot迭代器是迭代器中最简单的形式,支持在数据集上迭代一次不需要初始化。One-shoto处理大多数的基于队列的输入pipeline,但是他们不支持参数化。用Dataset.range()作为例子:
\begin{python}
dataset = tf.contrib.data.Dataset.range(100)
iterator = dataset.make_one_shot_iterator()
next_element = iterator.get_next()
for i in range(10):
    value = sess.run(next_element)
    assert i == value
\end{python}
initializable迭代器要求你使用前明确的运行iterator.initializer操作。作为不方便的交换,它允许你送入一个或者更多的tf.placeholder() tensor初始化你的迭代器,继续用Dataset.range():
\begin{python}
max_value = tf.placeholder(tf.int64, shape=[])
dataset = tf.contrib.data.Dataset.range(max_value)
iterator = dataset.make_initializable_iterator()
next_element = iterator.get_next()
# Initialize an iterator over a dataset with 10 elements.
sess.run(iterator.initializer, feed_dict={max_value: 10})
for i in range(10):
    value = sess.run(next_element)
    assert i == value
# Initialize the same iterator over a dataset with 100 elements.
sess.run(iterator.initializer, feed_dict={max_value: 100})
for i in range(100):
    value = sess.run(next_element)
    assert i == value
\end{python}
一个reinitializable迭代器可以通过多个不同的Dataset对象初始化。例如你可续有一个用随机扰动输入图像提高泛化性的输入pipeline一个验证输入pipeline在没有修改的数据上评价预测。这些pipeline将用有相同结构(每个组件有相同的类型和兼容的形状)的不同的Dataset对象
\begin{python}
# Define training and validation datasets with the same structure.
training_dataset = tf.contrib.data.Dataset.range(100).map(
    lambda x: x + tf.random_uniform([], -10, 10, tf.int64))
validation_dataset = tf.contrib.data.Dataset.range(50)
# A reinitializable iterator is defined by its structure. We could use the
# `output\_types` and `output\_shapes` properties of either `training\_dataset`
# or `validation\_dataset` here, because they are compatible.
iterator = Iterator.from_structure(training_dataset.output_types,
training_dataset.output_shapes)
next_element = iterator.get_next()
training_init_op = iterator.make_initializer(training_dataset)
validation_init_op = iterator.make_initializer(validation_dataset)
# Run 20 epochs in which the training dataset is traversed, followed by the
# validation dataset.
for _ in range(20):
# Initialize an iterator over the training dataset.
    sess.run(training_init_op)
    for _ in range(100):
        sess.run(next_element)
    # Initialize an iterator over the validation dataset.
    sess.run(validation_init_op)
    for _ in range(50):
        sess.run(next_element)
\end{python}
一个feedable迭代器可以和tf.placeholder用在一起来调用tf.Session.run时选择什么咧带起通过deed\_dict机制。它提供相同的函数作为重新初始化迭代器,但是当你在不同数据及切换的时候不要求你从数据集开始初始化。例如用相同的训练验证样本你可以用tf.contrib.data.Iterator.from\_string\_handle定义一个feedable迭代器允许你在不同的数据集之间切换:
\begin{python}
# Define training and validation datasets with the same structure.
training_dataset = tf.contrib.data.Dataset.range(100).map(
    lambda x: x + tf.random_uniform([], -10, 10, tf.int64)).repeat()
validation_dataset = tf.contrib.data.Dataset.range(50)
# A feedable iterator is defined by a handle placeholder and its structure. We
# could use the `output\_types` and `output\_shapes` properties of either
# `training\_dataset` or `validation\_dataset` here, because they have
# identical structure.
handle = tf.placeholder(tf.string, shape=[])
iterator = tf.contrib.data.Iterator.from_string_handle(
        handle, training_dataset.output_types, training_dataset.output_shapes)
next_element = iterator.get_next()
# You can use feedable iterators with a variety of different kinds of iterator
# (such as one-shot and initializable iterators).
training_iterator = training_dataset.make_one_shot_iterator()
validation_iterator = validation_dataset.make_initializable_iterator()
# The `Iterator.string\_handle()` method returns a tensor that can be evaluated
# and used to feed the `handle` placeholder.
training_handle = sess.run(training_iterator.string_handle())
validation_handle = sess.run(validation_iterator.string_handle())
# Loop forever, alternating between training and validation.
while True:
# Run 200 steps using the training dataset. Note that the training dataset is
# infinite, and we resume from where we left off in the previous `while` loop
# iteration.
    for _ in range(200):
        sess.run(next_element, feed_dict={handle: training_handle})
    # Run one pass over the validation dataset.
    sess.run(validation_iterator.initializer)
    for _ in range(50):
        sess.run(next_element, feed_dict={handle: validation_handle})
\end{python}
\subsection{消耗迭代器的值}
Iterator.get\_next()方法返回一个或多个迭代器的下一个元素tf.Tensor对象。每次迭代器被计算的时候他们得到数据集中的下一个元素,在TensorFlow中调用Iterator.get\_next()不会立即前进迭代器。相反你需要在一个TensorFlow表达式返回tf.Tensor独享,传递表达式的结果给tf.Session.run得到表达式的结果个下一个迭代器。
如果迭代器到大数据的尾部,执行Iterator.get\_next()操作将报出tf.errors.OutOfRangeError。这个点后迭代器将进入不稳定状态,你必须再次初始化它:
\begin{python}
dataset = tf.contrib.data.Dataset.range(5)
iterator = dataset.make_initializable_iterator()
next_element = iterator.get_next()
# Typically `result` will be the output of a model, or an optimizer's
# training operation.
result = tf.add(next_element, next_element)
sess.run(iterator.initializer)
print(sess.run(result))  # ==> "0"
print(sess.run(result))  # ==> "2"
print(sess.run(result))  # ==> "4"
print(sess.run(result))  # ==> "6"
print(sess.run(result))  # ==> "8"
try:
    sess.run(result)
except tf.errors.OutOfRangeError:
    print("End of dataset")  # ==> "End of dataset"
\end{python}
一个常用的模板是创建一个try-except块的训练循环包装器:
\begin{python}
sess.run(iterator.initializer)
while True:
    try:
        sess.run(result)
    except tf.errors.OutOfRangeError:
        break
\end{python}
如果数据集中的每个元素都有迭代的结构在相同的迭代结果下Iterator.get\_next()将返回一个或者更多的tf.Tensor。
\begin{python}
dataset1 = tf.contrib.data.Dataset.from_tensor_slices(tf.random_uniform([4, 10]))
dataset2 = tf.contrib.data.Dataset.from_tensor_slices((tf.random_uniform([4]), tf.random_uniform([4, 100])))
dataset3 = tf.contrib.data.Dataset.zip((dataset1, dataset2))
iterator = dataset3.make_initializable_iterator()
sess.run(iterator.initializer)
next1, (next2, next3) = iterator.get_next()
\end{python}
注意计算任何next1,next2或者next3将对所有的组件前进迭代器。一个典型的消耗迭代器将不能包含在单个表达式中的所有组件。
\subsection{读输入数据}
\subsubsection{消耗NumPy数据}
如果你的所有数据都适合于存储,一个简单的方法是用Dataset.from\_tensor\_slices()转换他们为tf.Tensor对象创建一个Dadaset。
\begin{python}
# Load the training data into two NumPy arrays, for example using `np.load()`.
with np.load("/var/data/training_data.npy") as data:
    features = data["features"]
    labels = data["labels"]
# Assume that each row of `features` corresponds to the same row as `labels`.
assert features.shape[0] == labels.shape[0]
dataset = tf.contrib.data.Dataset.from_tensor_slices((features, labels))
\end{python}
注意上面的带吧将在你的TensorFlow图中创建一个嵌入的features和labels作为一个tf.constant()操作。对于小的数据集这是很有用的,但是比较浪费存储,因为数据的内容将被多次复制可能达到tf.GraphDef protocal buffer的2GB限制。
\begin{python}
# Load the training data into two NumPy arrays, for example using `np.load()`.
with np.load("/var/data/training_data.npy") as data:
    features = data["features"]
    labels = data["labels"]
    # Assume that each row of `features` corresponds to the same row as `labels`.
assert features.shape[0] == labels.shape[0]
features_placeholder = tf.placeholder(features.dtype, features.shape)
labels_placeholder = tf.placeholder(labels.dtype, labels.shape)
dataset = tf.contrib.data.Dataset.from_tensor_slices((features_placeholder, labels_placeholder))
# [Other transformations on `dataset`...]
dataset = ...
iterator = dataset.make_initializable_iterator()
sess.run(iterator.initializer, feed_dict={features_placeholder: features,
                                        labels_placeholder: labels})
\end{python}
\subsection{消耗TFRecord数据}
一些数据集有一个或者多个文件。tf.contrib.data.TextLineDataset提供了一个简单的方法从一个或者更多text文件提取
行给定一个或者更多的文件名TextLineDataset将产生一个或者更多的字符串值元素。向TFRecordDataset,TextLineDataset接受ffilenames作为一个tf.Tenspor,因此你可以通过tf.placeholder参数化它
\begin{python}
filenames = ["/var/data/file1.txt", "/var/data/file2.txt"]
dataset = tf.contrib.data.TextLineDataset(filenames)
\end{python}
默认情况下一个TextLineDataset产生文件的每一行,这也许并不是你想要的,例如一个文件的开投行有一些注释。这些行可以用Dataset.skip()移除和Dataset.filter()转换。为了应用这些转换在每个分割的文件,我们用Dataset.flat\_map()为每个文件创建一个迭代的Dataset
\begin{python}
filenames = ["/var/data/file1.txt", "/var/data/file2.txt"]

dataset = tf.contrib.data.Dataset.from_tensor_slices(filenames)

# Use `Dataset.flat\_map()` to transform each file as a separate nested dataset,
# and then concatenate their contents sequentially into a single "flat" dataset.
# * Skip the first line (header row).
# * Filter out lines beginning with "#" (comments).
dataset = dataset.flat_map(
    lambda filename: (
        tf.contrib.data.TextLineDataset(filename)
        .skip(1)
        .filter(lambda line: tf.not_equal(tf.substr(line, 0, 1), "#"))))
\end{python}
\subsection{用Dataset.map()处理数据}
Dataset.map(f)通过使用函数f作用于输入数据集的每个元素生成一个新的数据集。它通过函数编程语言用map函数应用到列表。这个函数f接受tf.tensor对象代表一个单个的输入元素,返回一个代表一个数据集中单个元素tf.Tensor对象。它通过标准的TensorFlow操作转化一个元素为另一个。
\subsection{解析tf.Example protocal buffer消息}
一些输入的pipeline从TFRecord格式的文件提取tf.train.Example protocal buffer消息,用tf.python\_io.TFRecordWriter。每个tf.train.Example记录包含一个或者多个特征,输入pipline通常转换这些特征为tensor。
\begin{python}
# Transforms a scalar string `example\_proto` into a pair of a scalar string and
# a scalar integer, representing an image and its label, respectively.
def _parse_function(example_proto):
    features = {"image": tf.FixedLenFeature((), tf.string, default_value=""),
                "label": tf.FixedLenFeature((), tf.int32, default_value=0)}
    parsed_features = tf.parse_single_example(example_proto, features)
    return parsed_features["image"], parsed_features["label"]

# Creates a dataset that reads all of the examples from two files, and extracts
# the image and label features.
filenames = ["/var/data/file1.tfrecord", "/var/data/file2.tfrecord"]
dataset = tf.contrib.data.TFRecordDataset(filenames)
dataset = dataset.map(_parse_function)
\end{python}
\subsection{解码图像数据变换大小}
当在一个真实世界的图像数据中训练一个神经网络,需要转换不同大小到一个同样的大小,因此他们也许处理为一个固定的尺寸
\begin{python}
# Reads an image from a file, decodes it into a dense tensor, and resizes it
# to a fixed shape.
def _parse_function(filename, label):
    image_string = tf.read_file(filename)
    image_decoded = tf.image.decode_image(image_string)
    image_resized = tf.image.resize_images(image_decoded, [28, 28])
    return image_resized, label

# A vector of filenames.
filenames = tf.constant(["/var/data/image1.jpg", "/var/data/image2.jpg", ...])

# `labels[i]` is the label for the image in `filenames[i].
labels = tf.constant([0, 37, ...])
dataset = tf.contrib.data.Dataset.from\_tensor\_slices((filenames, labels))
dataset = dataset.map(\_parse\_function)
\end{python}
\subsection{用专门的Python logic}
考虑到性能要求,我们鼓励你尽可能用TensorFlow操作处理你的数据。然而当解析你的数据时有是有调用额外的python操作处理数据是有用的。为了这么做,在Dataset.map()转换中调用tf.py\_func()操作
\begin{python}
import cv2

# Use a custom OpenCV function to read the image, instead of the standard
# TensorFlow `tf.read\_file()` operation.
def _read_py_function(filename, label):
  image_decoded = cv2.imread(image_string, cv2.IMREAD_GRAYSCALE)
  return image_decoded, label

# Use standard TensorFlow operations to resize the image to a fixed shape.
def _resize_function(image_decoded, label):
  image_decoded.set_shape([None, None, None])
  image_resized = tf.image.resize_images(image_decoded, [28, 28])
  return image_resized, label

filenames = ["/var/data/image1.jpg", "/var/data/image2.jpg", ...]
labels = [0, 37, 29, 1, ...]

dataset = tf.contrib.data.Dataset.from_tensor_slices((filenames, labels))
dataset = dataset.map(
    lambda filename, label: tf.py_func(
        _read_py_function, [filename, label], [tf.uint8, label.dtype]))
dataset = dataset.map(_resize_function)
\end{python}
\subsection{简单的批处理}
一个最简单的批处理是堆叠数据集中n个连续的元素。Dataset.batch()变换就是这么做的,和tf.stac()k一样应用元素的每个组件,每个组件i所有的元素必须有一个相同的形状tensor。
\begin{python}
inc_dataset = tf.contrib.data.Dataset.range(100)
dec_dataset = tf.contrib.data.Dataset.range(0, -100, -1)
dataset = tf.contrib.data.Dataset.zip((inc_dataset, dec_dataset))
batched_dataset = dataset.batch(4)

iterator = batched_dataset.make_one_shot_iterator()
next_element = iterator.get_next()

print(sess.run(next_element))  # ==> ([0, 1, 2,   3],   [ 0, -1,  -2,  -3])
print(sess.run(next_element))  # ==> ([4, 5, 6,   7],   [-4, -5,  -6,  -7])
print(sess.run(next_element))  # ==> ([8, 9, 10, 11],   [-8, -9, -10, -11])
\end{python}
\subsection{批量的tensorpadding}
上面的方法要要求所有的元素有相同的尺寸,然而一些模型(sequence模型)中输入数据有不同的形状。为了处理这些情况,Dataset.padded\_batch()使你通过制定一个或者更多维(需要padding)转换不同形状的tensor为一个batch。
\begin{python}
dataset = tf.contrib.data.Dataset.range(100)
dataset = dataset.map(lambda x: tf.fill([tf.cast(x, tf.int32)], x))
dataset = dataset.padded_batch(4, padded_shapes=[None])

iterator = dataset.make_one_shot_iterator()
next_element = iterator.get_next()

print(sess.run(next_element))  # ==> [[0, 0, 0], [1, 0, 0], [2, 2, 0], [3, 3, 3]]
print(sess.run(next_element))  # ==> [[4, 4, 4, 4, 0, 0, 0],
                               #      [5, 5, 5, 5, 5, 0, 0],
                               #      [6, 6, 6, 6, 6, 6, 0],
                               #      [7, 7, 7, 7, 7, 7, 7]]
\end{python}
Dataset.padded\_batch()转换允许你为每组件的一维度设置不同的pading,他柯旭有变化的长度或者固定的长度,他可以覆盖padding的值(默认为0)。
\subsection{处理多epoch}
Dataset API提供了两个主要的方法处理相同数据的多个epochs,最简单的方法是用Dataset.repeat()变换数据集在多个epoch、例如,在输入中创建一个数据集10 epochs
\begin{python}
filenames = ["/var/data/file1.tfrecord", "/var/data/file2.tfrecord"]
dataset = tf.contrib.data.TFRecordDataset(filenames)
dataset = dataset.map(...)
dataset = dataset.repeat(10)
dataset = dataset.batch(32)
\end{python}
使用Dataset.repeat()变换没有参数重复输入将不确定。Dataset.repeat()变换连接他的参数每个epochs没有任何结束信号和下一个epoch的开始信号。如果你想接收每个epoch信号,你可以写一个循环在数据集的末尾捕获tf.errors.OutOgRange。
\begin{python}
filenames = ["/var/data/file1.tfrecord", "/var/data/file2.tfrecord"]
dataset = tf.contrib.data.TFRecordDataset(filenames)
dataset = dataset.map(...)
dataset = dataset.batch(32)
iterator = dataset.make_initializable_iterator()
next_element = iterator.get_next()

# Compute for 100 epochs.
for _ in range(100):
    sess.run(iterator.initializer)
    while True:
    try:
        sess.run(next_element)
    except tf.errors.OutOfRangeError:
        break

  # [Perform end-of-epoch calculations here.]
\end{python}
\subsection{随机打乱输入数据}
Dataset.shuffle()用和tf.RandomShuffleQueue方法随即打乱输入数据集,它保持一个固定的buffer平均的从bugger选择下一个元素:
\begin{python}
filenames = ["/var/data/file1.tfrecord", "/var/data/file2.tfrecord"]
dataset = tf.contrib.data.TFRecordDataset(filenames)
dataset = dataset.map(...)
dataset = dataset.shuffle(buffer_size=10000)
dataset = dataset.batch(32)
dataset = dataset.repeat()
\end{python}
\subsection{用高级APIs}
tf.train.MonitoredTrainingSession API简化了一些在分布式方面运行方面的处设置。MonidotedTrainingSession用tf.errors.outOfRangeError作为训练完成的标记,因此用Dataset API,我们推荐用Dataset.make\_one\_shot\_iterator()例如:
\begin{python}
filenames = ["/var/data/file1.tfrecord", "/var/data/file2.tfrecord"]
dataset = tf.contrib.data.TFRecordDataset(filenames)
dataset = dataset.map(...)
dataset = dataset.shuffle(buffer\_size=10000)
dataset = dataset.batch(32)
dataset = dataset.repeat(num_epochs)
iterator = dataset.make_one_shot_iterator()

next_example, next_label = iterator.get_next()
loss = model_function(next_example, next_label)

training_op = tf.train.AdagradOptimizer(...).minimize(loss)

with tf.train.MonitoredTrainingSession(...) as sess:
    while not sess.should_stop():
        sess.run(training_op)
\end{python}
为了在tf.estimator.Estimator的input\_fn中使用一个Dataset,我们推荐用Dataset.make\_one\_shot\_iterator(),例如:
\begin{python}
def dataset_input_fn():
    filenames = ["/var/data/file1.tfrecord", "/var/data/file2.tfrecord"]
    dataset = tf.contrib.data.TFRecordDataset(filenames)

# Use `tf.parse\_single\_example()` to extract data from a `tf.Example`
# protocol buffer, and perform any additional per-record preprocessing.
    def parser(record):
        keys_to_features = {
            "image_data": tf.FixedLenFeature((), tf.string, default_value=""),
            "date_time": tf.FixedLenFeature((), tf.int64, default_value=""),
            "label": tf.FixedLenFeature((), tf.int64,
                                    default_value=tf.zeros([], dtype=tf.int64)),
    }
        parsed = tf.parse_single_example(record, keys_to_features)

        # Perform additional preprocessing on the parsed data.
        image = tf.decode_jpeg(parsed["image_data"])
        image = tf.reshape(image, [299, 299, 1])
        label = tf.cast(parsed["label"], tf.int32)

        return {"image_data": image, "date_time": parsed["date_time"]}, label

  # Use `Dataset.map()` to build a pair of a feature dictionary and a label 
  # tensor for each example.
  dataset = dataset.map(parser)
  dataset = dataset.shuffle(buffer_size=10000)
  dataset = dataset.batch(32)
  dataset = dataset.repeat(num_epochs)
  iterator = dataset.make_one_shot_iterator()

  # `features` is a dictionary in which each value is a batch of values for
  # that feature; `labels` is a batch of labels.
  features, labels = iterator.get_next()
  return features, labels
\end{python}
