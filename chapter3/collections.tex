\section{collections}
collections是Python内建的一个集合模块,提供了许多有用的集合。
\subsection{namedtuple}
namedtuple是一个函数,用来创建一个自定义的tuple对象,并且规定了tuple元素的个数,并且可以用属性而不是索引访问tuple的某个元素。
定义一个Point类型。
\begin{python}
from collections import namedtuple
Point = namedtuple('Point',['x'.'y'])
p = Point(1,2)
p.x
p.y
\end{python}
上面创建的Point是tuple的一种子类:
\begin{python}
	>>>isinstance(p,Point)
	True
	>>>isinstance(p,tuple)
	True
\end{python}
用坐标和半径表示一个元,可以用namedtuple定义:
\begin{python}
	Circle = namedtuple('Circle',['x','y','r'])
\end{python}
\subsection{deque}
deque是为了高效实现插入和删除操作的双向列表,适合与队列和栈:
\begin{python}
In [1]: from collections import deque
In [2]: q = deque(['a','b','c','d'])
In [3]: q.append('f')
In [4]: q
Out[4]: deque(['a', 'b', 'c', 'd', 'f'])
In [5]: q.appendleft('1')
In [6]: q
Out[6]: deque(['1', 'a', 'b', 'c', 'd', 'f'])
In [7]: q.rotate()
In [8]: q
Out[8]: deque(['f', '1', 'a', 'b', 'c', 'd'])
\end{python}
\subsection{defaultdict}
使用dict时,如果key不存在就会抛出KeyError。如果不希望key不存在时,返回会一个默认值,就可以采用defaultdict。
\begin{python}
In [9]: from collections import defaultdict
In [10]: aa = defaultdict(lambda:'N/A')
In [11]: aa['key1'] = 'abc'
In [12]: aa['key1']
Out[12]: 'abc'
In [13]: aa['key2']
Out[13]: 'N/A'
\end{python}
除了key不存在时返回默认值外,defaultdict的其他行为和dict完全一样。
\subsection{OrderdDict}
使用dict时,Key是无序的。对dict做迭代时,我们无法确定Key的顺序。要保持key的顺序可以使用OrderDict:
\begin{python}
In [1]: from collections import OrderedDict
In [2]: a = dict([('a',1),('E',2),('B',3),('Q',5)])
In [3]: a
Out[3]: {'B': 3, 'E': 2, 'Q': 5, 'a': 1}
In [4]: b = OrderedDict([('a',1),('E',2),('B',2),('Q',5)])
In [5]: b
Out[5]: OrderedDict([('a', 1), ('E', 2), ('B', 2), ('Q', 5)])
\end{python}
OrderedDict的Key会按照插入的顺序排列,不是key本身的排序。
\begin{python}
In [7]: od['z'] = 1
In [8]: od['y'] = 2
In [9]: od['x'] = 3
In [10]: od
Out[10]: OrderedDict([('z', 1), ('y', 2), ('x', 3)])
\end{python}
OrderedDict可以实现FIFO的dict,当容量超出限制时,先删除最早添加的Key。
\begin{python}
from collections import OrderedDict
class LastUpdatedOrderedDict(OrderedDict):
    def __init__(self, capacity):
        super(LastUpdatedOrderedDict, self).__init__()
        self._capacity = capacity
    def __setitem__(self, key, value):
        containsKey = 1 if key in self else 0
        if len(self) - containsKey >= self._capacity:
            last = self.popitem(last=False)
            print('remove:', last)
        if containsKey:
            del self[key]
            print('set:', (key, value))
        else:
            print('add:', (key, value))
            OrderedDict.__setitem__(self, key, value)
\end{python}
\subsection{Counter}
Counter是一个简单的计数器,例如统计字符出现的个数。
\begin{python}
In [11]: from collections import Counter
In [12]: c = Counter()
In [13]: for ch in 'programming':
...:     c[ch]=c[ch]+1
...:     
In [14]: c
Out[14]: Counter({'a': 1, 'g': 2, 'i': 1, 'm': 2, 'n': 1, 'o': 1, 'p': 1, 'r': 2})
\end{python}
