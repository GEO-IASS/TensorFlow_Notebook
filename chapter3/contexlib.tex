\section{contextlib}
在Python中,读写文件时必须在使用完成后正确关闭它们。正确的关闭文件资源的方法是使用try\ldots finally
\begin{python}
try:
    f = open('/path/to/file','r')
    f.read()
finally:
    if f:
	f.close()
\end{python}
写try\ldots finally非常繁琐。Python的with语句允许我们非常方便的使用资源,而不必担心没有关闭,所以上面的代码可以简化为:
\begin{python}
with open('path/to/file','r') as f:
    f.read()
\end{python}
并不是只有open()函数返回的fp对象才能使用with语句。实际上,任何对象,只要实现了正确的上下文管理就能用于with语句。

实现上下文管理通过\_\_enter\_\_和\_\_exit\_\_这两个方法实现。例如,下面的class实现了这种方法:
\begin{python}
class Query(object):
    def __init__(self,name):
        self.name = name
    def __enter__(self):
        print('begin')
        return self
    def __exit__(self,exc_type,exc_value,traceback):
        if exc_type:
            print('Error')
        else:
            print('End')
    def query(self):
        print('Query info about%s...'%self.name)
with Query('Bob') as q:
    q.query()
\end{python}
编写\_\_enter\_\_和\_\_exit\_\_仍然很繁琐,因此Python的标准库contextlib提供了更简单的写法,上面的代码可改写为如下:
\begin{python}

from contextlib import contextmanager
class Query(object):
    def __init__(self,name):
        self.name = name
    def query(self):
        print('Query info about%s...'%self.name)
@contextmanager
def create_query(name):
    print('Begin')
    q = Query(name)
    yield q
    print('End')

with create_query('Bob') as q:
    q.query()
\end{python}
@contexmanager 这个decorator接受一个generator,用yield语句把with\ldots as var把变量输出出去,然后,with语句就能正常工作的。很多时候,我们希望在某段代码执行前后自动执行特定代码,可以用@contexmanager实现。例如:
\begin{python}
@contexmanager
def tag(name):
    print("<%s>"%name)
    yeild 
    print("</%s>"%name)
with tag("h1"):
    print("hello")
    print("world")
\end{python}
上述代码的执行结果为:
\begin{python}
<h1>
hello
world
</h1>
\end{python}
代码的执行顺序为:
\begin{enumerate}
	\item with语句首先执行yield之前的语句,因此打印出<h1>;
	\item yield调用会执行with语句内所有语句,因此打印出hello和word。
	\item 最后执行yield之后的语句,打印出</h1>
\end{enumerate}
因此\@contexmanager让我们通过编写generator简化上下文管理。
\@closing
如果一个对象没有实现上下文,我们就不能把它用于with语句。这时候可以调用closeing()吧该对象变为上下文对象。例如,用with语句使用urlopen():
\begin{python}
from contextlib import closing
from urllib.request import urlopen
with closing(urlopen('https://www.python.org')) as page:
    for line in page:
        print(line)
\end{python}
closing也是一个经过\@contextmanage装饰的generator,这个generator编写起来其实十分简单:
\begin{python}
@contexmanager
def closing(thing):
    try:
        yield thing
    finally:
        thing.close()
\end{python}
