\chapter{requests}
\section{快速上手}
\subsection{发送请求}
\begin{python}
import requests
r = requests.get('https://github.com.timeline.json')
r = reques
\end{python}
\subsection{requests库的7个主要方法}
\begin{center}
\begin{tabular}{|p{3cm}|p{12cm}|}
\hline
requests.request()&够找一个请求,支持一下各方法的基础方法\\
\hline
requests.get()&获取HTML网易的主要方法,对应于HTTP的GET\\
\hline
requests.head()&获取HTML网页头信息的方法,对应于HTTP的HEAD\\
\hline
requests.post()&向HTML网页提交POST请求的方法,对应于HTTP的POST\\
\hline
requests.put()&像HTML网页提交PUT请求的方法,对应于HTTP的PUT\\
\hline
requests.patch()&像HTML网页提交局部修改请求,对应于HTMP的PATCH\\
\hline
requests.delete()&像HTML网页提交删除请求,对应于HTTP的DELETE\\
\hline
\end{tabular}
\end{center}
requests.get(url,params=None,**kwargs)
\begin{itemize}
\item url:想要获取的网页的url链接。
\item params:url中额外的参数,字典或字节流格式,可选
\item **kwargs:12个控制访问的参数。
\end{itemize}
\subsection{request对象的属性}
\begin{center}
\begin{tabular}{|p{3.5cm}|p{12cm}|}
\hline
属性&说明\\
\hline
r.status\_code&HTTP请求的返回状态,200表示连接成功,404表示失败\\
\hline
r.text&HTTP响应内容的字符串形式,即,url对应的页面内容\\
\hline
r.encoding&从HTTP header中猜测响应的内容编码方式\\
\hline
r.apparent\_encoding&从内容分析出的响应内容编码方式(备选编码方式)\\
\hline
r.content&HTTP响应内容的二进制形式\\
\hline
\end{tabular}
\end{center}
\subsection{理解encoding和apparent\_encoding}
r.encoding:从HTTp header中猜测的响应内容编码方式,如果header中不存在charset,则认为编码为ISO-8859-1 r.text根据r.encoding显示网页内容\newline
r.apparent\_encoding:根据网页内容分析出的编码方式,可以看做是r.encoding的备选。\newline
\subsection{理解Requests库的异常}
\begin{center}
\begin{tabular}{|p{4.5cm}|p{12cm}|}
\hline
异常&说明\\
\hline
requests.ConnectionError&网络链接错误异常,如DNS查询时白,拒绝链接\\
\hline
requests.HEEPError&HTTP错误异常\\
\hline
requests.URLRequired&URL缺失异常\\
\hline
requests.TooManyRedirects&超过最大重定向次数,产生重定向异常\\
\hline
requests.ConnectTimeout&连接远程服务器超时异常\\
\hline
requests.Timeout&请求URL超时,产生超时异常\\
\hline
requests.raise\_for\_status()&如果不是200,产生异常,requests.HTTPError\\
\hline
\end{tabular}
\end{center}
r.raise\_for\_status()方法在内部判断r.status\_code是否等于200,不需要额外家if语句,该语句便于利用try-except进行异常处理。
\subsection{HTTP协议}
HTTP:Hypertext Transfer Protocol 超文本传输协议。\newline
HTTP是一句"请求与响应"模式的,武装台的应用层协议,HTTP协议采用URL定位网络资源的标志,URL格式如下:\newline
\begin{center}
http://host[:port][path]\\
host:合法的Internet主机域名和IP地址。\\
port:端口号,缺省端口为80\\
path:请求资源的路径\\
\end{center}
{\centering{\textbf{HTTP协议对资源的操作}}}
\begin{center}
\begin{tabular}{|p{3cm}|p{12cm}|}
\hline
方法&说明\\
\hline
GET&请求获取url位置的资源\\
\hline
HEAD&请求获取URL位置资源的响应消息报告,即获得该资源的头部信息\\
\hline
POST&请求像URL位置的资源后附加新的数据\\
\hline
PUT&请求URL位置存储一个资源,覆盖原URL位置的资源\\
\hline
PATCH&请求局部更新URL位置的资源,即改变该处资源的部分内容\\
\hline
DELETE&请求删除URL位置存储的资源\\
\hline
\end{tabular}
\end{center}
head方法的使用
\begin{python}
In [3]: r = requests.head('http://www.baidu.com')
In [4]: r
Out[4]: <Response [200]>
In [5]: r.headers
Out[5]: {'Server': 'bfe/1.0.8.18', 'Date': 'Tue, 08 Aug 2017 11:46:32 GMT', 'Content-Type': 'text/html', 'Last-Modified': 'Mon, 13 Jun 2016 02:50:04 GMT', 'Connection': 'Keep-Alive', 'Cache-Control': 'private, no-cache, no-store, proxy-revalidate, no-transform', 'Pragma': 'no-cache', 'Content-Encoding': 'gzip'}
\end{python}
post方法的使用(像URLPOST一个表单,自动编码为form)
\begin{python}
In [10]: payload = {'key1':'value1','key2':'value2'}
In [11]: r = requests.post('http://httpbin.org/post',data=payload)
In [12]: print(r.text)
{...
 "form":{"key2":"value2",
         "key1":"value1"
 }.
}
\end{python}
put方法
\begin{python}
In [18]: r = requests.put('http://httpbin.org/put',data = payload)

In [19]: print(r.text)
{
  "args": {}, 
  "data": "", 
  "files": {}, 
  "form": {
    "key1": "value1", 
    "key2": "value2"
  }, 
  "headers": {
    "Accept": "*/*", 
    "Accept-Encoding": "gzip, deflate", 
    "Connection": "close", 
    "Content-Length": "23", 
    "Content-Type": "application/x-www-form-urlencoded", 
    "Host": "httpbin.org", 
    "User-Agent": "python-requests/2.18.3"
  }, 
  "json": null, 
  "origin": "210.47.0.232", 
  "url": "http://httpbin.org/put"
}
\end{python}
requests.request(method,url,**kwargs)\newline
{\centering{\textbf{**kwargs控制访问参数}}}
\begin{center}
\begin{tabular}{|p{3cm}|p{12cm}|}
\hline
params&字典或字节序列,作为参数增加到url中\\
\hline
data&字典,字节序列或文件对象,作为Request的内容\\
\hline
json&JSON格式的数据,作为Request的内容\\
\hline
header&字典,HTTP定制头\\
\hline
cookies&字典或CookieJar,Request中的cokkie\\
\hline
auth&元组,支持HTTP认证功能\\
\hline
file&字典类型,传输文件\\
\hline
timeout&设定草食时间,s为单位。\\
\hline
proxies&字典类型,设定访问代理服务器,可以增加登录认证\\
\hline
allow\_redirects&:True/False,默认为True,重定向开关\\
\hline
stream&True/False,默认为True,获取内容立即下载开关\\
\hline
verify&True/False,默认为True,认证SSL整数开关\\
\hline
cert&本地SSL整数路径\\
\hline
\end{tabular}
\end{center}
