\chapter{Tensorflow API}
\section{tf.squeeze}
tf.squeeze(input,axis=None,name=None,squeeze\_dims=None)
说明:从指定的Tensor中移除1维度。
\begin{itemize}
\item input:tensor,输入Tensor。
\item axis:列表,指定需要移除的位置的列表,默认为空列表[],索引从0开始squeeze不为1的索引会报错。
\item name:caozuoide名字
\item squeeze\_dims:否决当前轴的参数。
\item 返回一个Tensor,形状和input相同,包含和input相同的数据,但是不包含有1的元素。
\item 异常:squeeze\_dims和axis同时指定时会有ValueError。
\end{itemize}
\# 't' is a tensor of shape [1, 2, 1, 3, 1, 1]\newline
shape(squeeze(t)) ==> [2, 3]\newline
\# 't' is a tensor of shape [1, 2, 1, 3, 1, 1]\newline
shape(squeeze(t, [2, 4])) ==> [1, 2, 3, 1]\newline
\section{tf.stack}
stack(values,axis=0,name='stack'):stack一个n维tensor为n+1维tensor。
给定一个长度为N的形状为(A,B,C)的tensor,如果axis==0输出tensor的形状为(N,A,B,C),如果axis==1,输出tensor的形状为(A,N,B,C)
\# 'x' is [1,4]\newline
\# 'y' is [3,6]\newline
\# 'z' is [3,6]\newline
stack([x,y,z])==>[[1,4],[2,5],[3,6]]\newline
stack(x,y,z,axis=1)==>[[1,2,3],[4,5,6]]\newline
tf.stack([x,y,z]) = np.asarray([x,y,z])\newline
参数:
\begin{itemize}
	\item 一个Tensor列表。
	\item 整数,默认为0,支持负坐标。
	\item 操作的名字。
	\item[S] 一个stack的Tensor。
	\item[S] ValueError:如果axis超过[-(R+1),R+1)
\end{itemize}
\textsl{Example}
\begin{python}
import tensorflow as tf
x = tf.constant([1,4])
y = tf.constant([2,5])
z = tf.constant([3,6])
r1 = tf.stack([x,y,z])
r2 = tf.stack([x,y,z],axis=1)
with tf.Session() as sess:
    print(sess.run(r1).shape)
    print(sess.run(r2).shape)
\end{python}
\section{tf.metrics}
accuracy(labels,predictions,weights=none,metrics\_collections,updates\_collections=none,name=none)
\begin{itemize}
	\item labels:tensor,和predictions的形状相同,代表真实值。
	\item predictions:tensor,代表预测值。
	\item weights:tensor,rank可以为0或者labels的rank,必须能和label广播(所有的维度必须是1,或者和labels维度相同)
	\item metrics\_collection:accuracy应该被增加的一个collectiobn列表选项。
	\item update\_collections:update\_op应该添加的选项列表。
	\item name:variable\_scope名字选项。
	\item accuracy:返回值tensor,代表精度,总共预测对的和总数的商。
	\item update\_op:返回值适当增加total和count变量和accuracy匹配。
	\item valueerror:异常如果predictions和labels有不同的形状,或者weight不是none它的形状不合prediction匹配,或者metrics\_collections会哦这updates\_collections不是一个list或者tuple。
\end{itemize}
\section{tf.reshape}
tf.reshape(tensor,shape,name=None)
\begin{itemize}
	\item Tensor:一个Tensor。
	\item shape:一个列表,数值类型为int32或者时int64
	\item name:操作的名字。
	\item[S] 指定形状的Tensor。
\end{itemize}
\begin{python}
import tensorflow as tf
a = tf.linspace(0.,9.,10)
b = tf.reshape(a,[2,5])
with tf.Session() as sess:
    a = sess.run(a)
    b = sess.run(b)
print(a.shape)
print(b.shape)
\end{python}
\section{tf.app.flags}
\subsection{DEFINE\_boolean}
DEFINE\_boolean(flag\_name,default\_value,docstring):定义一个'boolean'类型的flag。
\begin{itemize}
\item flag\_name:flag的名字,是一个字符串。
\item default\_value:flag应被被看作一个boolean的默认值。
\item docstring:用flag的一个帮助信息。
\end{itemize}
\subsection{DEFINE\_boolean}:定义一个'boolean'类型的flag。
\begin{itemize}
\item flag\_name:flag的名字,是一个字符串。
\item flag\_default\_value:默认的boolean类型的值。
\item docstring:用flag的一个有用的帮助信息。
\end{itemize}
\subsection{DEFINE\_float}:定义一个浮点数类型的flag。
\begin{itemize}
\item flag\_name:作为flag的名字,应该是字符串。
\item default\_value:flag的默认值,应该是浮点数。
\item docstring:用flag的一个有用的帮助信息。
\end{itemize}
\subsection{DEFINE\_integer}:定义一个整数的flag。
\begin{itemize}
\item flag\_name:flag的名字,应该是字符串。
\item default\_value:flag的默认值,应该是一个整数。
\item docstring:用flag的一个有用的帮助信息。
\end{itemize}
\subsection{DEFINE\_string}:定义一个字符串的flag。
\begin{itemize}
\item flag的名字,应该是字符串。
\item default\_value:flag的默认值,应该是字符串。
\item docstring:用flag的一个有用的帮助信息。
\end{itemize}

